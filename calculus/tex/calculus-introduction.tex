\section{What is calculus?}
Pratically speaking, calculus is the gateway into higher mathematics for first-year college students. It is widely required for freshman college students in STEM fields, and very often for student in close fields. An introduction to calculus is often viewed as an initiation in mathematical rigor, as a first exposure to mathematical methods which go beyond the high school level.

In some high schools, introduction to calculus is expected. In other high schools, it is not. Eiehter way, calculus is, in the United States at least, an essential level of introduction to mathematics and mathematical methods.


\section{What are the topics in calculus?}
One of the first things that a student asks of a couse is, ``what am I supposed to learn here?'' Here, we will give a brief and purposefully simplistic overview of the topics usually learned in a standard calculus course. Here they are:
\begin{enumerate}
\item Limits
\item Derivatives
\item Integrals
\item Series
\item Differential equations
\end{enumerate}

\section{Why are these topics of calculus studied?}
As with most fields, the reason for the sequence of topics taught is partly historical and is partly practical. Calculus has been taught in a certain way for centuries, for better or for worse; in parallel or because of this, depending on your perspective, calculus is an introduction into the mathematical mehtods of many fields, and has therefore become a ``fundamental'' course for very many fields.

Regardless of how we got here, the methods that one learns in calculus have become essential to many, many fields, and therefore it is empiricaly accepted as a standard fundamental course.

\section{How is calculus usually taught?}
Calculus is often taught as an extension of high school classes: ``Memorize this, regurgitate it thus.'' This is a symptom of poor mathematics teachers. It also causes many students to feel as though they have to memorize a vast amount of material. Unfortunately, this vast amount of material that supposedly needs to be memorized - does not need to be memorized. There is a handful of principles which, if learned, will alow you to solve any basic calculus problem, if you learn instead of memorizing.

\section{How should calculus be taught?}
In the author's view, calculus should be taught as a way of thinking, not as a way of rotely memorizing how to manipulate certain cymbols in order to get a 10/10 of homework....

\section{How should I learn calculus?}
You should learn calculus not as a memorization of ``x is this, y is that, and teacher told me to the do this.'' If you unforunately happen to have a teacher who teachers you to memorize, you should understand that you have a teacher who does not truly understand what he or she is supposed to teach; you should understand that there is a reason that humanity came up with calculus, that there is a very good reason for which it was worth it for thoudsands of people to study this. 

This is not the equivalent of ``coloring by numbers.'' It is not supposed to be mechanical. There is something to understnad here; even if it turns out that you are not ultimately interested in mathematics, you will still gain from understanding and applying the concepts in calculus, as opposed to memorizing and regurgitating - ``coloring by numbers.''

\section{What is the best way for me to approach the study of calculus?}
The best way to approach the study if calculus is the same way you might approach any new field of study. For example....

\section{How should I use this companion text alongside my study of calculus?}
Your textbooks focus on teaching a student how to perform calculations. The intention of this book is to explain why you are studying what you are studing, and how it will be useful to you in the future. You should see this text as supplementary to your mandated course readings. You are being told what to read, as part of a standard curriculum. This book tries to illulstrate why it's worth studying these topics in and of themsleves, without having to, and therefore being honest about it.
